\documentclass[final,letterpaper,twoside,12pt,twocolumn]{report}
% if you use "report", you get a seperate title page
%\documentclass[final,letterpaper,twoside,12pt]{article}
%

\usepackage[utf8]{inputenc} 
%\usepackage[swedish]{babel}
\usepackage{fancyhdr}
\usepackage{graphicx}
\usepackage{empheq}

\begin{document}
\author{Vilhelm~Jutvik \thanks{To everyone at SICS and the NES group}}
\date{\today}
%\title{The adaption of the IKEv2 protocol for The Internet of Things MS Thesis project plan}
\title{IPsec for the Contiki OS}

\maketitle

% 
% en beskrivning av det problem som angripits: problemformulering och problemstrukturering;
% en beskrivning av de metoder som använts för att lösa problemet;
% tolkning och diskussion av resultat - en analys av hur väl problemet lösts;
% om problemet ej lösts fullständigt, en beskrivning av de troliga orsakerna till detta - även negativa resultat och erfarenheter kan ofta vara värda att rapportera (så slipper andra göra om det);
% innehållsförteckning;
% litteraturförteckning.
% Vidare skall
% 
% de egna metoderna granskas kritiskt;
% en jämförelse med andras arbeten rörande liknande problem ingå.

\begin{abstract}
In this the work the author explores...
\end{abstract}

\tableofcontents

\chapter{Populärvetenskaplig sammanfattning}
nödvändig bakgrundsinformation för en allmänbildad läsare som inte nödvändigtvis är specialist inom området;

hot and cold mediums

\chapter{Introduction}
The Internet has revolutionized communication. The author would like to argue that this happened because it drastically lowered the cost of communication by removing interface barriers, unifying standards and organizations. Internet is cheap while legacy, purpose-made, communication networks are expensive (e.g. consider the PSTN\footnote{Public Switched Telephone Network, your ordinary phone}). These lowered costs allowed people to send and receive information at an enormous rate, creating new types of medias and services such as the blog, the social network, the search engine etc. Because of these benefits, we try to expand the Internet, and now strides are being made to connect the objects that we surround ourselves with, creating what's commonly called he `Internet of Things' (abbreviated as IoT). It's a vision of an extended Internet where household objects (radiators, lighting etc), sensors and actuators in industrial machinery, cars etc are put `online'. Data is primarily sent and received by radio as cables are expensive to install and maintain.

The underlying motivation as to why we want to communicate with things are that they and their surroundings matters to humans. A thing such as freezer can notify building management when its about to break down and notify building management before the food is spoiled. Data from the vibration sensor in the bridge helps the engineer to model its health and plan maintenance. A thing can also help another thing, e.g. the temperature sensor telling the radiator to turn as the temperature is falling.

% This is of importance to humans as its believed that the IoT will greatly reduce the cost for communication thing to thing and man to thing. This enables more information to be shared, and more information usually translates into a better understanding of the environment, which translates into a more efficient world. An example of this would be an Internet-connected temperature sensor in a room which gives the IoT-enabled radiators a better idea about when and how much heat to apply.

Unlike a regular host on the Internet (e.g. a PC) an IoT device is often purpose-made for a few single tasks (e.g. measuring the temperature in the example above). Data is usually sent and received over radio. This allows the computer's electronics to be small, cheap and consume little power, so little that it can be run of a non-rechargeable battery for the system's entire lifetime. This set of technological characteristics are what commonly characterize the IoT.

%The promise of cheap, ubiquitous and easily installable computers bundled with the Internet's ease of communication is what fuels the vision of an Internet of Things.

Contiki is small and resource efficient operating system that tries to address the above requirements. With only 35 kB of ROM and 8 kB RAM it provides an IP stack as well as multitasking. On the multitude of platforms officially supported, the hardware is often composed of a small 16-bit CPU, a small non-rechargeable battery and a radio operating in the 2.4 GHz ISM band FIXCORRECT. Developed mainly at SICS, the OS have gathered a large following around the world, only surpassed by its relative TinyOS. Naturally, Contiki is the primary research platform for IoT at SICS, and is why the work of this thesis is based upon it.

As the IoT will control and monitor sensors as well as machines in our surroundings, security is a natural concern. Internet (more specifically, the IP protocol) by itself does not offer any security guarantees by default. This is especially true in the IoT world where the physical layer often is provided by radio. Therefore, this thesis will explore the possibilities of implementing the IPsec security extension of the IP-protocol in the Contiki OS. This will bring methods to secure communications to the IoT, that are also compatible with vast parts of the Internet.


%Although this allows communication with the rest of the Internet, it cannot be done in privacy. Any host in the path of a message between it and any other host can read, stop or manipulate the data. This is problematic if the IoT device measures or controls something deemed important to humans, such as a door lock or a motion sensor in a burglar alarm. Therefore the author have investigated whether or no it's feasible to implement support for an Internet standard that enables hosts to communicate securely over the Internet; IPsec. IPsec is an IETF\footnote{Internet Engineering Task Force} standard that is a part of IPv6, the coming replacement to the current IP protocol. This thesis outlines its design, implementation and evaluation of this standard and its utility in the Contiki OS.



%  For example;  As the hosts (computers) usually are located inside, or in the direct vicinity of the object of interest,    power is usually supplied by a small non-rechargeable battery and communicates by radio.\\
% \\
%  This implies that said devices must be able to use the same protocols and standards as the rest of the Internet. One such piece is the suite of security features offered by the IP protocol, called IPsec.\\
% \\
% Contiki is an OS for devices running on the Internet of Things. It features an implementation of the IP protocol and several other Internet standards. The purpose of this work is to add yet another standard to it; IPsec. IPsec is an extension to the IP protocol that enables hosts to communicate securely over the Internet. Naturally, there are several implementations of IPsec readily avaible, but we can't use them since they don't fit. IoT devices that runs Contiki are usually very constrained in memory as well as CPU, but above all, energy is most important. The typical IoT -device is battery powered. Because of this, the result of the work must be able to save resources whereever possible, 

\section{Problem statement}
As the IoT will control and monitor sensors as well as machines in our surroundings, security is a natural concern. Internet (more specifically, the IP protocol) by itself does not offer any security guarantees by default. This is especially true in the IoT world where the physical layer often is provided by radio. Any device with suitable radio equipment can read IP-packets destined for others, create and send packets while imposing as another host and manipulate data in transmissions destined for somebody else. This is problematic if the IoT device measures or controls something deemed important to humans, such as a door lock or a motion sensor in a burglar alarm. When receiving a message in the case of the former, the host controlling the door lock wants to assert that: the identity of sender is correct; that the message have not been tampered with; and preferably, it should be encrypted so that no other host except the addressee can read it. At the same time, the technology enabling this must be supported by the lock (the IoT device) and the other host, which might be any type of machine anywhere on the Internet. IPsec is one of many similar standards designed to solve these problems.

The difficulty of the work lies in making an implementation that is compatible with other hosts on the Internet (enabling communication), fulfilling the security requirements (correctness) while simultaneous assuring that the code is small enough to fit in the limited memory. This also relates to energy requirements, especially in the context of cryptography which is heavily used by IPsec, as computations consumes a sizable part of an IoT host's energy budget.

\section{Research question}
Can IPsec and IKEv2 be implemented within the current hardware boundaries while still being interopable with other Internet hosts?
 
%There are several different solutions to these problems. One of them is IPsec, which is the standard that the author has chosen to implement for the Contiki OS. The main hurdle 
%Solutions to these problems are provided by several different stan
%These problems are solved by several different standards
%The author's hypothesis is that the IPsec extension of the IP protocol can fulfill these criteria. 

%\section{Hypothesis / Proposed solution}
%The author has worked with the hypothesis that the aforementioned requirements can be fulfilled by the IPsec standard. It's a well estbalished protocol and most of the hosts connected to the Internet already has supported it for a number of years. The purpose of investigating whether or not it's feasible to solve this problem by implementing an existing Internet standard in the Contiki OS.


%implement support for an Internet standard that enables hosts to communicate securely over the Internet; IPsec. It's an extension of the replacement for the current IPv4 Internet Protocol, IPv6. The 

%listen in the network can send For example, consider the case of the door lock that is connected to the Internet via IoT. Anyone device which  When it receives a message telling it to unlock it wants to assert the identity of the sender, know 

% The goal of this thesis is the investigate the feasibility of using IPsec in the Contiki operating system. This implies that the software implementing the standard should not only fit on the device's limited ROM memory, but be demonstrated to actually do something useful.

% IPsec is an Internet standard for secure host-to-host communication. Some of its benefits have been mentioned above, but can it be implemented in Contiki?
%will endow Contiki with secure communication facilities, but at what cost, if it's possible at all? 

    %    Kan IPsec / IKE implementeras i Contiki på den här begränsade hårdvaran?
    % -> Vad krävs för att contiki-implementation / min implmenettation ska anses vara användbar?
    %          * Bevisa att den kan upprätta och hantera IKE -anslutningar
    %          * Minnes och beräkningskraft
					
\section{Method}
Much of the research in IoT is carried out by experiments because of the highly applied nature of the field. Indeed, most of the Internet as-of-today, was constructed on the basis of principles surmised from experiments. Therefore, the author decided to  implement the various components that makes up a functional IPsec system and then test them. The tests were carried out by subjecting the system to common communication scenarios with other IPsec-enabled Internet hosts: handshaking a new secure connection; receiving and transmitting packets with various security policies applied; housekeeping of connections. The success or failure of the test is determined if it worked, or if it succeeded, how well it did so. Finally, memory consumption and cpu time is measured for the different tasks. It's the results of these tests that will form the basis of the evaluation of the 1) quality and workmanship of the implementation 2) suitability of IPsec for IoT in general.

\subsection{Development process}
The source code was managed using the Git source code revision system. Development was organized in such a way that the project was run in one branch\footnote{Git branches are similar in concept to those in other SCMS. ...} (hereafter called \emph{develop}) parallell to Contiki's master branch. Important patches / changes that occurred in Contiki while the development was underway could thus simply be fetched / merged into \emph{develop}. Patches in develop that proved  was then merged in from the 

\section{Limitations}
Firstly, the purpose of this work is to answer the questions raised in the problem statement, not to create an implementation that's ready for production use. Thus, parts of the standard that are not necessary to meet this end can be omitted. This also includes parts that are labeled as REQUIRED in the standard documents, as they're only necessary if you strive towards full compliance, which in terms of features arguably is superset of that of interoperability.

\section{Alternative Approaches}
There are many ways of securing communication in a computer network, far too many to review in this section. What follows is a swift review of the closest alternatives for security in IoT networks. A thorough review and comparison will be made of these in the discussion towards the end of this thesis.

The traditional approach in the IoT world has been to secure communication by using the IEEE's 802.11.4 link layer and its security features. There are also plenty of research articles outlining completely new security schemes, but none of these have made it into and IETF standard as far as the author is aware. Another alternative is the IETF's TLS protocol\footnote{The TLS protocol is a successor to SSL} that operates next to the application layer, but no implementation of that has been completed as of today to the best of the author's knowledge.

A final possibility would be to come to a conclusion by using analysis. Supposing that the main obstacle to a functional solution is that of ROM, RAM, CPU speed and energy requirements. As one can assume the cryptographic libraries to consume the majority of the resources, one can arrive at an approximation for the hardware requirements by benchmarking them stand-alone. This would certainly help in arriving at an upper bound for the hardware requirements, but the analysis would not tell us anything about the operational problems and benefits that a complete implementation would.

%These can be ordered by the network layer\footnote{The notion of network layers is derived from the idealized OSI-model in which network protocols are ordered in a stack of seven layers.} at which they operate.

%\paragraph{The link layer} can provide authentication, confidentiality and integrity if using the IEEE 802.15.4 radio link layer, commonly employed in IoT systems. The benefits are that it's transparent to 
	
%	Alternative approaches: Beräkna storleken på kryptobibliotek mm + skatta kodkraven på andra delar, summera sedan
%	Scientific contributions: Evaluerin
%	Report structure

\section{Scientific Contributions}
This report makes three scientific contributions. The first is the demonstration of that it's feasible to implement IPsec with dynamic key management and IKEv2 in the Contiki OS. The second is that it's possible to simplify the IPsec protocol in IPv6-IoT environments without sacrificing much in interoperability. The final is that IPsec's network-centric policy language is ill suited for ad-hoc, self-organized network environments, such as that created by the RPL\footnote{The Routing Protocol for Low-Power and Lossy Networks (RPL) is an IETF standard (RFC 6553) that creates dynamic routing topologies} routing protocol commonly used in conjunction with IoT.

\section{Report Structure}
The purpose of the chapter \emph{Introduction} chapter is to give the reader an overview of the report. The next chapter, \emph{Background}, will begin by a discussion of the problems that are particular to the field of IoT and how Contiki is constructed. This will be followed by a brief recount of the background and design of IPsec and 802.11.4's security features. In \emph{Design and Implementation} the author outlines how the IPsec standard is adapted to IoT and Contiki. Simplifications and omissions of features are explained and argued for. The design is then evaluated in \emph{Evaluation} by measuring parameters such as time, energy and memory usage when sending and receiving data. Space will also be given to observations of a qualitative nature, such as how well the standard integrates with the Contiki OS. The author's opinions of the evaluation and suggestions for future work is given in \emph{Conclusion}.


Background
Implementation
Evaluation
Discussion
Conclusions and Future Work

\chapter{Background}
The purpose of this chapter is to introduce the reader to the Contiki OS and the IPsec standard. A good understanding of these systems are necessary to understand the later chapter `Design and Implementation'.

\section{Contiki}
Stuff

\section{IPsec}
Secures IP

\section{IKEv2}
Handshakes for IPsec

\chapter{Design and Implementation}
\section{Architecture}
IPsec and IKE communicates through the databases SPD and SAD. The PAD doesn't need to be a database.

The IPsec subsystem is implemented inside the uIP6 stack.

[BIG image outlining what information is passed between the IP-stack, the IPsec processing subsystem, IKEv2 and the databases (SPD, SAD)]

\section{IPsec}
We implement IPsec by inserting hooks into the uIP-stack.

\section{IKEv2}
The IKEv2 service is implemented as a Contiki Process. Each session is modelled as a mealy state machine [img].

The machine is built to be extended. The following is implemented as of now:


\chapter{Evaluation}

\section{Memory requirements}
Maximum stack and heap requirements
Per SA requirement

\section{Latency}
* IKE handshake: Energy and time. Total setup time.
* IPsec packet: Energy required for receiving, energy required for transmitting. Roundtrip times.

\chapter{Improvements}



\end{document}

% SLUT!


\section{IoT}

SICS\footnote{Swedish Institute of Computer Science} have been conducting research the field of IoT for over a decade. The purpose is to explore, and try to solve, the problems in the software stack. One of these areas is end-to-end security between a host in the IoT network and one on the Internet.

 This has resulted in the very small event driven operating system Contiki. 

Why we want IoT.
In order to bring IoT out of the lab and into the society it needs to be secure. A pair of hosts in the network must be able to authenticate each other and protect their communication from privy eyes by means of encryption. The solution concerned also needs to be standards-compliant with the rest of the Internet.

This suggests that TLS or IPsec may be suitable.

TLS secures a TCP connection. IPsec secures everything from the network layer and up. The purpose of this work is to implement IPsec and IKEv2 

The topic of this work is to implement IPsec and IKEv2 with the purpose of investigating the following:


\subsection{The Contiki OS}%\footnote{Internet of Things}}
Contiki is an event based operating system...

\subsection{IPsec}
IPsec is an IETF\footnote{Internet Engineering Task Force} standard defined in RFC ...

\subsection{TLS}
TLS is a ...

\section{Scope}

The purpose of this thesis is to explore the IPsec venue; is it a feasible solution?

\subsection{Motivation}

The goal of the thesis is to write an implementation of IPsec and IKEv2 in order to prove that the following is possible:
* Low power, resource constrained hardware can utilize ``heavy'' Internet-standards and asymmetric encryption
* The implementation should be able to communicate with a wide range of systems

From this, the following should be required in order to form a valid proof-of-concept:
* The implementation should be polite towards other hosts. It should be able to handle a wide range of situations.
* Where time does not allow for a full implementation of a certain functionality, it should be clear that the implementation can be extended without significant rewrites of other parts.

The original scope was the implementation of the IKEv2 service as described in RFC 5996 [...] for the Contiki OS. After a more thourough investigation this was found not to be feasible.


\chapter{Design}
The purpose of the
The original design goals requirements of the implementation were
1) It should implement a subset of IPsec and IKEv2 where feasible, or 
2) 
chose
choose
chosen

\section{Constraints}
Memory, processing, energy...


\chapter{Implementation}
GCC, utveckling på PC-hårdvara
Hanterverket
(kort avsnitt)


\chapter{Evaluation}
Evaluation of the impl. in relation to the design goals


\section{Performance and requirements}
Processing and memory requirements

\section{Design evaluation}
Implications of design choices



\end{document}
